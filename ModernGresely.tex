\PassOptionsToPackage{unicode=true}{hyperref} % options for packages loaded elsewhere
% article example for classicthesis.sty
\documentclass[12pt,letter]{article} % KOMA-Script article scrartcl
\usepackage{lipsum}
\usepackage{url}

% Preamble additions for Gresely
\usepackage{marginnote}
\usepackage[utf8]{inputenc}
\usepackage[T1]{fontenc}
\usepackage[normalem]{ulem}
\usepackage{tablefootnote}

% save typing for the red text colouring
\newcommand{\red}[1]{\textcolor{red}{#1}}
% This abstracts the insertion of the Gresely source page numbers

%% use this one for double-left notes
\newcommand{
\srcpgL}[1]{\noindent{\hspace{-1.85in}\color{Gray}{\rule[0.5ex]{6.0in}{1pt}~#1}}\\
}
%% use this one for left-right notes
\newcommand{
\srcpgR}[1]{\noindent{\hspace{-0.65in}\color{Gray}{\rule[0.5ex]{5.5in}{1pt}~#1}}\\
}
\newcommand{\srcpg}[1]{\srcpgL{#1}}

% left oriented margin notes
\newcommand{\tmnL}[1]{% Text Margin Note Left
\reversemarginpar\marginparsep=11mm{\marginnote{#1}}
}
\newcommand{\tmnR}[1]{% Text Margin Note Right
\normalmarginpar\marginnote{#1}
}
\newcommand{\tmn}[1]{
\tmnL{#1}
}

\newcommand{\nmn}[1]{% Numeric Margin Note
\reversemarginpar\marginparsep=3mm\marginnote{#1}
}
% (Not) End of Gresely specific preable additions

\usepackage[nochapters,dottedtoc]{classicthesis} % For some font and style options
\RequirePackage[letterpaper, top=1in, bottom=1.5in, 
  left=1.5in, right=1.5in,showframe=false]{geometry}

%More Gresely Formatting
%% use these for left-left formatting
\marginparwidth=1.5in
\oddsidemargin=0in
\textwidth=6.5in

%% use these for left/right formatting.  Comment out for left-left
%\marginparwidth=1in 
%\oddsidemargin=1.15in 
%\textwidth=3.25in 

\begin{document}
    \title{\rmfamily\normalfont\spacedallcaps{the gresley manuscript}}
    \author{\spacedlowsmallcaps{accadamia della danza}}
    \date{} % no date
    
    \maketitle
    
    \begin{abstract}
        \noindent\lipsum[1] Just a test.\footnote{This is a footnote.}
    \end{abstract}
    \tableofcontents
\newpage
\section{Introduction}
\subsection{About the Notebook}
The existence of the Gresley Manuscript was brought to the attention of the dance community through David Fallows’ 1999 article\footnote{Fallows, David. “The Gresley Dance Collection, c.1500.” Royal Musical Association Research Chronicle, no. 29 (1996): 1–20. Available online from both Taylor and Francis\\  http://www.tandfonline.com/toc/rrmc20/29/1?nav=tocList and JStor http://www.jstor.org/stable/25099451?seq=1}.  This includes a historical introduction, and transcriptions of both the text and the music.  The transcription is not completely diplomatic (an academic term meaning “matching the original in both content and form as closely as possible”), and it is covered by the journal’s copyright.

The source is referred to as the “Gresley manuscript” because it is part of the Gresley collection in the Derbyshire Records Office in Matlock, England. The author might have been associated with the family of Gresley of Drakelow, or the book could have come into their possession at a later point.  It has been dated to circa 1500\footnote{Fallows. p.3}.
\subsection{Goals of the Project}
One of our aims in creating our own transcription was to have one that was open for reuse and included critical apparatus to explain our editorial decisions.  In the course of the project we discovered several minor issues with Fallows’ transcription, which we have hopefully amended without adding further errors.

We put forth some interpretation and discussion, but our main goal in this paper is to provide a transcription and also normalization of the words and music notes. We agree with the Fallows text in most cases, and we would like to provide more information useful for reconstruction. We would like to make the information contained within the manuscript more available.

\subsection{Codicology}
Fallows provides a codicological analysis of the notebook in his article, so what follows is just a summary.  The entire notebook is 90 pages, a mix of vellum and paper, with outside dimensions of 4.75x3.75 inches.  There are two sets of page numbers in the MS, one set in the middle of the page in a circle, and a later one at the bottom outer corner of the page, both in pencil and both page, not folio, numbers.  Following Fallows and O’Sullivan\footnote{Banys, John, John of Seville, and Philip of Tripoli. “John Banys’s Dance Manuscript.” Commonplace Book. Derby, Derbyshire, UK, c 1490. Derbyshire Records Office.  See Curatorial notes by Margaret O’Sullivan} we used the second set. There are several sections of text:
\begin{itemize}
\item 1-33: A text on Chiromancy (palm reading), taken from The John of Seville Chiromancy, in Latin.
\item 34: blank
\item 35-50: A text on physiognomy (the reading of faces).  This Latin version, called the Secreta Secretorum, was translated by Philip of Tripoli. 
\item 51-79: Dances, in English, as below
\begin{itemize}
\item 51-53: List of Titles in two columns
\item 54-66: Dance choreography, part 1
\item 67-72: Dances, part 2.  The same hand but a different color of ink
\item 73-79: Music
\end{itemize}
\item 80-85: Latin prayers
\item 86-90: blank
\end{itemize}
The text of the dances is in a block 3.15x2.25 inches, with the outer 0.15 (approximate) inch devoted to the titles and marginal numbers.  The block varies from 17-19 lines per page and although there is no visible ruling, the blocks are consistent and well justified.  In the List of Titles, however the lines are not consistent between columns.   The List of Titles, first section of dances (pp. 54-66), and music are done in a dark brown ink.  The start of the dance description is usually indicated with a red pilcrow sign, and often the start of a new clause begins with a red stroke.  These appear to have been done after the main text was completed, with the red ink over the brown ink of the main text. The second section of dances (pp. 67-72) lacks any red highlighting or pilcrow marks, so the dances start only with a capital letter.  In this section the ink is either a darker brown or black and appears to have been drier; there are many letters without complete coverage and many of the letters have a patchy appearance.  Outside the text block are titles (often set-off with a border), dance set size (for two or three), and large arabic numerals (between 1 and 4, mostly 2 and 3) whose meaning is still unknown and which are missing from the second part.   

There is not a complete count of the quires available, but quires of 6 bifolia seem to have parchment for the outer and inner (ff.1,6,7,12), and the one described quire of 11 has an additional parchment bifolium at the 6th bifolium (ff. 1,6,11,12,17,22).  The paper leaves are water damaged and have been reinforced with a tissue-paper overlay, making any watermarks nearly invisible, but on pp. 45 and 51 a watermark in the shape of a Gothic “Y” is visible\footnote{O’Sullivan, ibid.}.  Similar watermarks are found in the second-half of the 15th century, most commonly in the 1470-80s\footnote{TO-DO Aaron to update}.  

\subsection{Copyright}
The Gresley Manuscript was created in England and currently resides in the UK and is subject to their copyright laws.  Additionally, as an unpublished work, it is governed by changes to the law that were enacted in 1988.  Prior to 1988 works in the UK were under copyright for 70 years post-mortem for works published by a living author and 50 years from publication for works published posthumously.  The copyright clock for unpublished works did not formally start, therefore these works were under copyright in perpetuity.  The 1988 law, which actually went into effect in 1989, provides a 50 year term for unpublished works, hence the copyright for all unpublished works of sufficient age will expire in 2039.  This created a class of orphaned works, works which are under copyright but where the owner is unknown or cannot be found.  For these works full publication is impossible until copyright expiration in 2039.

A 2014 update to the law carved out exemptions for “libraries, educational establishments and museums, as well as archives”\footnote{“EUR-Lex - 32012L0028 - EN - EUR-Lex,” accessed May 14, 2017,\\  http://eur-lex.europa.eu/legal-content/EN/TXT/?uri=CELEX\%3A32012L0028.} to make use of these orphaned works, including digitization and providing on-premise access.  It also created an Orphaned Work Licensing Scheme\footnote{“Copyright: Orphan Works - GOV.UK.” Accessed September 10, 2017.\\  https://www.gov.uk/guidance/copyright-orphan-works.} which provides for seven-year, non-exclusive licenses for these works, but only for publication in the UK.  

As an unpublished work, the Gresley Manuscript is under this law and copyright until 2039.  Although the Derbyshire Records Office has a complete digital facsimile, they cannot give or sell a copy of it.  We were able to get digital copies of pages 50-79, which include the List of Titles, the dances, the music and the final page of the Latin text on physiognomy, under the copyright allowance for “non-commercial research or private study”\footnote{“Exceptions to Copyright - GOV.UK.” Accessed September 10, 2017.\\  https://www.gov.uk/guidance/exceptions-to-copyright.}.  Under this exemption the Records Office was only willing to provide those pages and we are not able to distribute them either digitally or in print.  

\subsection{Purpose of the Manuscript}

There has been a great deal of theorizing about the purpose of this notebook, both in our group and beyond. We think it can be stated definitively that the purpose cannot be known for sure based solely on the manuscript itself. This is unfortunate, since the purpose clearly affects how to approach the descriptions and therefore it affects the reconstructions. It has been suggested that these are notes jotted for personal use while learning each dance. It has been suggested that these are notes or a draft for a possible future publication. There are likely more suggestions that we haven’t even heard. Herein we present our current theory, to add to the debate!

People in period made reference books for themselves; we refer to them now as “commonplace books.” Since there are several other topics covered in the same notebook, and since those notes are known to have been copied from other reference texts, it put us in mind of these kinds of personal reference books. 

This source is not someone's rough notes, it is an archive. The section begins with a list of titles, which is normally interpreted as a list of dances, but could instead be a list of music. Most of these do not have associated dance notes, and some are well-known tunes. This list is sorted by first letter, but neither the dance descriptions nor the music follow this order.   These descriptions are densely written; when one dance ends the next one begins with no whitespace separation, but there are relatively generous margins which contain the dance titles and other information. The first set of dance descriptions have marks in red ink that make the clause divisions easier to see. The music pages are less densely written than the descriptions. There are blank pages, so there was space for more dance descriptions or music if that had been wanted. 

Someone valued this information enough to go to some trouble and expense to make a reference work, and this was the format and content he wanted.

\subsection{Transcription Conventions}

We have tried to make this transcription as diplomatic as possible, representing the scribal product as close to the original as possible.  To that end, we followed the following conventions:
\begin{itemize}
    \item Symbols and abbreviations are expanded inside square brackets, eg. [con]trary
    \item Lacunae are represented with empty brackets, []
    \item Thorns are written with a modern, typographical thorn: þ
    \item Roman and Arabic numerals are preserved as written.  The long ‘i’ that frequently ends a string of digits has been transcribed as a ‘j’, eg. 4 as iiij
    \item The vertical slash that ends clauses has been indicated with a semicolon
    \item Many clauses start with a stroke of red over the black ink of the letter.  This has been represented with a red letter in the text
    \item The start of pages is indicated with a horizontal line in the text with the page number
    \item Capital letters are used as in the text
    \item Elevated letters in the text are represented as such with superscript, eg. “ye \textsuperscript{last} in” on p.54
    \item Superscript digits are editorial commentary footnotes
    \item Where scribal erasures can be read, they are written with strikethrough
    \item The arabic numbers that appear in the margins are copied, with our best effort made to properly align them.
    \item In the list of titles there is a small symbol to the right of many, but not all, of the titles.  This has been represented as such “*)”.  The meaning is unknown
\end{itemize}

\subsection{Scribe}

The legibility of the scribe’s handwriting has been unfairly impugned. The hand, which is very close and contains abbreviations, requires familiarization before it can be read easily, but the script is clear and well-executed. The use of abbreviations is not consistent, but the symbols are reliable in their meaning when they appear.

The spelling, on the other hand, is quite unreliable in a way that is typical for the period. Even when a term was spelled in one way most of the time, that was no guarantee that it was spelled that way all of the time. We do not believe that spelling variations have any bearing on the interpretations of the dance steps (eg. a “torn” is equivalent to a “torne” and they both mean “turn”). Similarly, the variations in phrasing choice (eg. “torne” vs “torne about”) do not appear to have separate meanings when it comes to dance steps, though that is harder to verify. We did make an effort to look into whether we could use phrasing variations to help us with reconstruction, and it was not successful. The author appears to sometimes want to add more clarifying words, and sometimes not. 

\subsection{Abbreviations \& extraneous marks}
TO-DO This whole subsections still needs to be handled.

\newpage 
\subsection{The Music}
\subsubsection{Overview of the Music}
The Gresley MS includes seven pages of music in landscape orientation, numbered 73 through 79.  Pages 73 through 75, which are paper, are badly faded with water damage and some areas bear only the faintest trace of the original contents.  The remaining four pages are parchment; these are in better condition and the ink is darker.  The music, like the text itself, being handwritten, does require some interpretation. Each page contains 3 staves, and it is not clear how many pieces of music they comprise. There are many more corrections and erasures in the music section than in the dance text. 

\subsubsection{Writing the Music}

The music is presented in 15th-century white mensural notation. It consists of a single melody line written on a five-line staff.  The lines of the staves appear to have been drawn with a straightedge, but they are not uniformly parallel or evenly spaced. Mensuration signs are not included.  Several C clefs are used; the music is mostly pitched in the tenor range.  Note values range from breve to semiminim, including dotted note values and one rest.  Note heads for minims and semiminims are triangular, and note-heads for semibreves are loosely diamond-shaped, but sometimes have a triangular appearance.  No ligatures are used.

Bar lines in the modern sense are absent, as was typical in this period.  Multiple vertical bars appear at the very end of several pieces as a kind of line filler or final flourish.  Single vertical bars are used in places to divide the music into sections.  For some of these sections, a numeral is written above the music; these numerals have been interpreted as repeat instructions.  We note that some of these marks, which have been transcribed in other articles as the numeral 1, resemble signa congruentiae.   Since these marks float above sections of music rather than being attached to specific notes, and because of their resemblance to other numerals in the manuscript, they are interpreted here as numerals. 

There is text written above and between the lines of music, and in two places on the staff itself.  Most pieces are given what appears to be a title, written above the beginning of the music in a hand very similar to that used for the dance choreographies.  On pages 73 and 74, some titles, and some sections of music, appear to have been struck out with horizontal strokes.  For the first piece on page 73, a new title is given after an obliterated one.  Some of the text which occurs between two lines of music has been represented here as being below the previous staff, rather than as a title above the next staff; these assignments should be considered open to interpretation.  

\subsubsection{Transcription of the music}

We have made our best effort to faithfully transcribe the original, since we have access to a high-quality enlarged color copy.  Our transcription tries to copy the marks in the original document with a minimum of interpretation. The transcription is presented first in mensural-style notation similar to that in the original manuscript, and then again with modern clefs and note-heads.

\subsubsection{The List of Titles: Table of Contents? Index? Neither?}
Although it appears at the start of the dance section, the List of Titles is by no means a modern table of contents. It has a tabular format, and the dances which are described are all listed therein. However, it is alphabetically organized, and contains no information regarding the order in which the dances appear thereafter. Also, it contains more than 3 times as many names as there are dances described later, and several of the described dances are listed by an alternate spelling. (There are 92 titles, and 26 dances described). The list is tidily written and ends well before the end of a page, and there are no corrections, erasures or added lines.  

All the musical pieces that have legible titles are included in the List of Titles. 

Considering that the list is alphabetical and more “complete” than the descriptive section, and there are no blank pages after the list, it seems likely that the list was prepared first, with the intent to fill in the descriptions subsequently. This bolsters the interpretation that these pages were written with intention to be a reference for later use, and not a collection of notes taken while learning to dance. 

In our experience, the List of Titles tended to cause more confusion rather than answering any of our questions about the correct names of dances, or their provenance.

\subsection{The Accademia and its Process}

The Carolingian Accademia della Danza was founded in 1992, as a complement to the long-running weekly local dance practice.  The practice was open and social, so it tended to avoid the harder dances and was not a good venue for research.

The Accademia della Danza was created as a place to focus on dance research and performance.  It has run on and off over the past 25 years, under a variety of leaders, on many projects.  We usually spend anywhere from 6 months to 2 years on a given project, delving deeply into it before moving on to the next subject.  Past projects have included sections on Domenico, the Basse Danse repertoire, Caroso, Ebreo, Pattricke/Lovelace and many other topics.

The Accademia has always been intensely collaborative.  We get together once every month or two, frequently starting with a meal, and work through the next section.  Our ethos is built around the idea that more eyes on the problem are usually better, and even the newest members can provide new insights.  We like to include a variety of skills -- the current group includes a musician, a scribe, and a specialist in manuscripts, as well as dance specialists.  Decisions are made by consensus whenever possible, and our meetings tend to be boisterous.

% TO-DO Fix Gundorm's name
Beginning in 2016, under the leadership of Lord Gun∂ormr Dengir, we began to work on a fresh transcription of the Gresley MS.  This was the first time we had tackled a principled transcription, but the process remained the same: we passed around the facsimile, and worked together to puzzle out what it said, often debating for ten minutes what a given word might say.  (Sometimes with recourse to the OED and other sources.) 

\subsubsection*{Participants}
\begin{itemize}
\item Aaron Macks - Gun∂ormr Dengir
\item Ailish Eklof - Ailís inghean Muirgen
\item Mark Waks - Justin du Coeur
\item Lisa Koch - Ysabel da Costa
\item Meredith Courtney - Mara Kolarova
\item Karen Veale - Thyra Eiriksdottir
\item Jesse Wertheimer - Hermankyn of Carolingia
\item Heather Cougar - Caterina Ginevra Beltrami
\item Jamin Brown - Alexandre Saint Pierre
\end{itemize}

\newpage 
\section{List of Dances}
\begin{center}
%\LARGE
\begin{table}[ht]
\Large
\begin{tabular}{lcl clr}   %remove | to remove vertical centre line
A & 2 & Avent{[}y{]} *& 2 & Colert rose & \\
  & 2 & Aleigemoy & 2 & Camamell & \\
  & 2 & Aras *) & 2 & Chaumby & \\
  & \underline{2} & \underline{Arandell} *) & 2 & Caricanto & \\
  & 3 & Assay *) & 3 & Conmfort *) & \\
  & 3 & Attendans & & Damysyn & D \\
  & 2 & Al þ[e] floe[rs] of & & Daysy dallya \tablefootnote{Unclear if this is a new dance or continuation of the name "Damysyn"} * & \\
  &   & the bro[m]m & 2 & Defformes ) & \\
  &   & Armen *) & 2 & Delyte *) & \\
  & 2 & Bellybrok & 2 & Desir *) & \\
  & 2 & Beaute *) & 2 & Eglyntyn & E \\
  & 2 & Basell *) & 2 & Ev[er] to end[]y & \\
  & 2 & Baeon *) & 3 & Eglamowr & \\
  & 2 & Beteyn *) & 3 & Esp[er]ans *) & \\
  & 2 & Burgon \tablefootnote{"ur" is speculative, minims are unclear}  *) & 3 & Egle *) & \\
  & 3 & Beugull *)  \\
  & 2 \& 3 & Bonryn & 3 & Fortune *) & F \\
C & 2 & Carbonet & 2 & Grene Gy[n]g[er] & \\
  & 2 & Crymesyn & 2 & Giffith*) & \\
\end{tabular}
\end{table}
\newpage
\begin{table}[ht]
\Large
\begin{tabular}{lcl clr}   %remove | to remove vertical centre line
  & 3 & Grenlene ) & 2 & Macomplent & \\
H & 2 & Hathorne *) & 2 & Malory*)  & \\
I & 2 & Joyes desiy[r] & 3 & Mounferramit & \\
  & 2 & Iohn bean= & 3 & Moubray & \\
  &   & [s]hir *) \\
  & 3 & Jeynys & • & New castell da[n]sh & N \\
  & 4 & Joly bokett ) & 3 & Northu[m]bland & \\
  & 2 & Ine *) & 2 & New founde & \\
K & • & Kendall *) & &  Newtythynge & \\
L & 2 & Lefarannt & 3 & New yere *) & \\
  & 2 & Lez mount[n]es & 2 & Orlyanse *) & \\
  & 2 & Lez novell & 3 & Orynge *) & \\
  & 2 & Leu[er] duy \tablefootnote{Exact reading unclear} & 2 & Peteous *) & P \\
  & 2 & L[][er]nell \tablefootnote{Second 2 letters unclear} *) & 2 & Plesantyne & \\
  & 2 & Lubens di= & 2 & Plesans *) & \\
  &   & stune[u] *) & 3 & Petygey & \\
  &   & Len[er]s a da[n]sh \tablefootnote{There is a flying symbol over the "n", meaning unclear.  Could be part of the title of the previous dance (Lu bens)} & 2 & P[er]nes in gre & \\
  & 3 & Laduches & 3 & P[er]nez in gard & \\
M & 2 & My lady m[a]y[e/s] \tablefootnote{Final word unclear} & 3 & P[] [u]lo plesowr \tablefootnote{Final word unclear} & \\
\end{tabular}
\end{table}
\newpage
\begin{table}[ht]
\Large
\begin{tabular}{lcl clr}   %remove | to remove vertical centre line
  &   & P[er]synam[or] da[n]sh & 3 & Violett & \\
  & 2 & Princytory &   & What þe lust & \\
R & 2 & Russett *) & 2 & What soev[er] & \\
  & 2 & Roty loly ioy &   & yo well & \\
  & 2 & Raynes & & & \\
  & 2 & Roye *) & & & \\
S & 2 & Sa[n]gue  & & & \\
  & 2 & Synkants & & & \\
  & 3 & Solacz & & & \\
  & 4 & St[a]ngisnolet \tablefootnote{Reading unclear} & & & \\
  & 3 & Shrympe & & & \\
  &   & Sofferancz & & & \\
T & 2 & Talbott & & & \\
  & 2 & Tam[r]ett & & & \\
  & 3 & Th[us] a nysyt & & & \\
  & 2 & Thus shal at be & & & \\
V &   & Unafo[rzg]  & & & \\
  &   & Avaunt da[n]sh & & & \\
\end{tabular}
\end{table}
\end{center}


\newpage 
    \section{Dances}
    \textwidth=3.5in
    \marginparwidth=2in
    \oddsidemargin=2in

    %\finalVersionString 
    \reversemarginpar
%    \raggedright
    \subsection{Esperans} \raggedright 
\srcpg{54}
\tmn{Esp{[}er{]}ans de 3{[}bus{]}}Al the vj si{[}n{]}glis w{[}ith{]} a trett; þen þe \\* 
fyrst man goo compas till he \\ 
\tmn{Trace}come behend whil þe medyll \\ 
retrett thre and þe last iij singlis \\ 
\nmn{3}and þe medil iij singlis levyng\\ 
the last on; \red{t}he left hand and \\
the last iij retrette thus þe m-\\
edill endyth before þe \textsuperscript{last} in þe m-\\
eddist; \red{a}nd the ferst behynd y{[}us{]} \\ 
dannce iij tymes callyng\\ 
ev{[}er{]}y man as he standdith; \red{A}ft{[}er{]} þe\\
\nmn{3}end of the t{[}ra{]}ce þe ferst iij fur-\\
th outward t{[}ur{]}nyng Ayen his \\ 
face; then þe  last co{[}n{]}tur hym \\ 
and the medill to þe fyrste \& \\
then þe first to his place; \red{y}en \\ 
the i last to the \nmn{1}medyll and þe  \\
\srcpg{55}
medyll to þe last mans  place; \red{t}he\\
\nmn{1}first and the last chance place whil \\ 
\nmn{1}the medyll tornyth; \red{A}l at onys re-\\
trett iij bake; bak al  at ons; \red{t}hen \\ 
\nmn{1}the first t{[}or{]}ne whyll þe last t{[}or{]}ne i{[}n{]} \\ 
in hys own place; \red{t}hen al toged{[}er{]}\\ 
thre furth;

    \subsection{Talbott}
\tmn{Talbott de 2{[}bus{]}}
\nmn{1}Aft{[}er{]} the hend of the \\ 
\nmn{1}trace trett retrett and dep{[}ar{]}t the \\ 
first furth right þe secn{[}o{]}d {[}con{]}tr{[}ar{]}y \\ 
\tmn{doble t{[}ra{]}ce}hyme and t{[}ur{]}ne face to face; \red{t}hen \\ 
\nmn{2}brawle at onys and then com to-\\ 
gydd{[}er{]}; \red{t}hen trett \& retrett togeds \\ 
\nmn{2}w{[}ith{]} iij singlis forth and chance \\
hande; on the same wyes Agayn \\ 
\nmn{1}then to ged{[}er{]} ij doblis ij rakis and \\ 
A t{[}ur{]}ne;

    \subsection{Mowbray}
\tmn{Mowbray\\ de 3bs}
\nmn{1}Ev{[}er{]}y man trett \& retrett \\ 
then þe first \& þe last turne outward \\ 
\tmn{with t{[}ra{]}ce}
\nmn{2}the medyll furth iij singlis; \red{t}hen \\
\srcpg{56} 
all trett and retrett w{[}ith{]} halfe a t{[}or{]}ne \\ 
Face to face then mett toged{[}er{]} þen \\
medill ent{[}er{]} w{[}ith{]} halfe a turne; \red{Þ}en þe \\
\nmn{4} ffrest iij singlis outhward and \\
the last od{[}er{]} iij {[}con{]}tr{[}ar{]}y hym þe m{[}e{]}dill \\ 
retrett þe same tyme; \red{t}hen the \\
First and þe last iij bake and þ[e] \\
medill iij furth and mett All \\ 
togeder; \red{t}hen iij forth w{[}it{]} halfe \\ 
A torne; \red{t}hen þe last iij singlis \\ 
outhwartt the first {[}con{]}tr{[}ar{]}y hym \\ 
the medill retrett þe same ty[me]; \\ 
\red{t}hen þe first and þe last iij \\
bak the medill come betwen \\ 
them; \red{t}hen half A torne All \\ at onys.



\subsection{Egle}
All to gedir trett \& \\ 
\tmn{{[}Egle{]}\\ de 3bs}retrett then iij singlis forth \\
\ \\
\srcpg{57}
then the medill torne into þe first \\ 
manys place befor hym; þe first in-\\
to þe medils manys place whil þe \\
\tmn{Trace 3}last Brawlyth; \red{t}hen trett and retrett \\
ayen and iij forth; \red{t}hen þe first \\
brawll as he standith whill þe \\
secn{[}o{]}d and the thred change place \\
both on the right shuld{[}er{]} þis dance \\
iij tymes and then ev{[}e{]}y ma{[}n{]} shal \\
be in hys Awne place;

\subsection{{[}unnamed{]}}
\nmn{3}Aft{[}er{]} the \\
end of the t{[}ra{]}ce ev{[}er{]}y ma{[}n{]} at onys \\
retrett to A t{[}ri{]}angle then the first \\
thouth oder two whyll þay chance \\
place then all beyng in a t{[}ri{]}angle \\
the first though as he com froo; \\
þen oder two chance places þem \\
benig all in A t{[}ri{]}angle then þ{[}e{]} first \\
\srcpg{58} 
A long brawll alone þe secn{[}o{]}d a \\
Flowr de lice whith iij retrette; \\
\red{t}hen þe last ma{[}n{]} trett \& retrett \& \\
torne

\subsection{Bugill}
\tmn{Bugill\\ de 3us}After the end of the t{[}ra{]}ce\\
ev{[}er{]}y ma{[}n{]} toged{[}er{]} two doblis þan \\
þe first and þe last fourth right \\
the medill {[}con{]}t{[}ra{]}ry hy{[}m{]} and torne \\
\tmn{Whith t{[}ra{]}ce}face to face met in to a t{[}ri{]}angle \\
wyse; \red{t}hen all toged{[}er{]} iv singlis \\
\nmn{3}compass; \red{t}hen þe last ma{[}n{]} throth \\
whill þe oder two {[}con{]}ter hym; \red{t}hen \\
all ronde iv singlis then come \\
all toged{[}er{]} and dep{[}ar{]}t w{[}it{]} A torn\\
\nmn{3}the d{[}er{]} þay come fro{[}m{]}; \red{t}hen the \\
medill throth while þe first \& \\
the last chance place then \\
mett All togedere; \red{t}hen e{[}ve{]}ry ma{[}n{]} \\
\srcpg{59} 
\nmn{1}From od{[}er{]} retrett iij; \red{t}hen torn All\\
\nmn{1}at onys than all at onys trett \& ret{[}re{]}tt\\
then þe medill tourney þe first\\
aboute and leve hym on his left\\
hand whill þe last torne in his\\
Awn{[}e{]} place;

\subsection{Prenes a Gard}
\tmn{P{[}re{]}nes a gard\\ de trib{[}us{]}}Af{[}er{]} the end of the\\
tr{[}ac{]}e the first ma{[}n{]} lepe þe secn{[}o{]}d\\
\nmn{3}lepe the iij thred torne; \red{t}hen\\
the last lepe þe secn{[}o{]}d lepe þe\\
\tmn{A doble tr{[}ac{]}e}first torne the first lepe þe last\\
lepe þe medill torne; \red{t}hen þe\\ 
\nmn{3}the medill forth iij singlis w{[}t{]}\\
halfe a torne whill þe first and the\\
last retrett; \red{t}hen the first\\
torne þ{[}e{]} last lepe; \red{t}hen þe me-\\
dill ma{[}n{]} throth whill þe first \&\\
the last change place and the\\
medill to his place Agayn; \red{t}hen\\
the first t{[}or{]}ne retrett \& rake \\
\srcpg{60}
\nmn{3} whill \^{} \textsuperscript{þe 2{[}d{]}} t{[}or{]}ne rake and retrett and þ{[}e{]}\\
third retret rake \& torne then Al\\
at onys a flour dilice \& com to-\\
geders;

\subsection{Pernes on Gre}
\tmn{Pernes\\ on gre de 2{[}o{]}bs}Trace forth right vj\\
singlis ather torne oþ{[}er{]} aboute \&\\
forth right vj singlis Ayen; Aft{[}er{]}\\
\tmn{The t{[}ra{]}ce}the end of þe trace rak both to-\\
\nmn{3}ged{[}er{]} and torne; \red{t}hen face to face\\
\nmn{2}vj singlis  ethir {[}con{]}t{[}ra{]}ry od{[}er{]} and iij\\
retrette Ayen; \red{t}hen A flowr deli\\
\nmn{1}of both at onys; \red{t}hen change place\\
and torne face to face; \red{t}hen A flowr\\
delice and come toged{[}er{]};

\subsection{Princitore}
\tmn{Princitore\\ de duob{[}u{]}}After þe\\
end of þe trace A longe torne both\\
at onys; \red{t}hen þe first ma{[}n{]} iij forth\\
\tmn{Whith t{[}ra{]}ce}and þe od{[}er{]} iij bak and then loke\\
ov{[}er{]} þe shuldyr the secn{[}o{]}d ma{[}n{]} þe\\
same whill þe first ma{[}n{]} folowith\\
hyme in the same forme w{[}t{]}\\
obeysaunce at þe last end thus \\
\srcpg{61}
doo thre tymes and at þe thred ty{[}m{]}\\
both retrett then ether pase odyr\\
two tymes \& torne; then ether\\
come to od{[}er{]}; \red{t}hen the last ma{[}n{]} trett\\
retrett and torne;

\subsection{Armyn}
\tmn{Armyn de 3{[}bus{]}}
\nmn{3}After the end\\
of þe trace the last iij bak þe med-\\
\tmn{A doble t{[}ra{]}ce}lis od{[}er{]} iij bak the first od{[}er{]} iij bake\\
\red{t}hen þe first meve þe secn{[}o{]}d half\\
\nmn{3}torne þe secn{[}o{]}d move \& þe last\\
half torne the last move \and þ{[}e{]}\\
first halfe torne; \red{t}hen þe last\\
\nmn{2}vi singlis forth þe meddist as\\
many to hym; \red{t}hen þe first trett\\
\nmn{1}retret and torne as he standith\\
\nmn{3} \red{t}hen brawle al at onys on way \&\\
ayen the od{[}er{]} way; \red{t}hen ev{[}er{]}y ma{[}n{]} at\\
onys change place; \red{t}hen þe last\\
\nmn{1}man thruth w{[}t{]} A torne whill þe\\
first torne the secn{[}o{]}d A bought

\subsection{Whatsoever You Will}
\tmn{What so eu{[}er{]} y\\ wyll de 2 b{[}us{]}}
\nmn{a}After the end of the t{[}ra{]}ce trett \\
\ \\
\srcpg{62}
\tmn{Doble t{[}ra{]}ce}\ \\
\nmn{2}\red{A}nd retrett to geders and thre for-\\
th forth w{[}ith{]} A step; \red{t}hen trett \& retrett\\
\nmn{1}to geder and both torne at onys;\\
\red{t}hen both forth to ged{[}er{]} w{[}ith{]} vi sing=\\
\nmn{2}lis and \red{t}hen change hands \red{t}hen\\
Forth w{[}ith{]} od{[}er{]} sex singlis \& change\\
\nmn{1}hands; \red{t}hen trett \& retrettt at onys\\
\nmn{1}then rake both at onys

\subsection{Petygay}
\tmn{Petygay de 3{[}bus{]}}After þe\\
end of the tr{[}ac{]}e ev{[}er{]}y ma{[}n{]} ii{[}i{]} singlis\\
after od{[}er{]}; \red{t}hen þe doble trace Agayn\\
\tmn{Doble t{[}ra{]}ce}then torne all at onys

\subsection{Tamrett}
\tmn{Ta{[}m{]}rett\\ De doubus}\red{A}fter\\
\nmn{2}the end of the trace trett \& retret\\
and iij forth to ged{[}er{]}; \red{t}hen trett\\
\& retrett and iij bake then ethir\\
\nmn{2}torne othir Abought \^{}ii; \red{t}hen brayl\\
\tmn{whith t{[}ra{]}ce} \nmn{1} ethir {[}con{]}t{[}ra{]}ry to od{[}er{]} ij tymes \red{\&} retrett\\
iij other fro od{[}er{]} \& then com toged{[}er{]}

\subsection{Green Gynger}
\tmn{Grengy{[}n{]}g{[}er{]}\\ de doub{[}us{]}}\\
After þe end of þe t{[}ra{]}ce rak both \\
\srcpg{63}
\tmn{Duble t{[}ra{]}ce w{[}ith{]}\\ a hertt in the\\ end}\ 
on way and iij the end t{[}or{]}ne bak to\\
\nmn{2}bak; \red{t}hen rake ayen bak to bake\\
and in the end torne face to face\\
\tmn{Cherwell\\ thy wyne}then iij singlis ethir {[}con{]}t{[}ra{]}ry od{[}er{]} \&\\
\nmn{3}thre bak Ayen; \red{t}hen ethir retrett\\
from \^{}odir iij singlis; \red{t}hen come toged{[}er{]}\\
\nmn{1}and mak a hertt Ayen

\subsection{Sofferance}
\tmn{Sofferance\\ de duob{[}us{]}}\red{A}fter\\
the end of þe trace the first iij\\
\nmn{3} forth and torne whyll þe last \\
\tmn{A doble t{[}ra{]}ce}retrett; \red{t}hen the last forth \&\\ 
torne whyll þe first retrett\\
and then both retrett ethir\\
\nmn{2}From od{[}er{]}; \red{t}hen the first A flowr\\
delice the sec{[}o{]}nd Anothir; \red{t}hen\\
\nmn{3}ethir {[}con{]}tr{[}a{]}ry other iij singlis on þ{[}e{]}\\
left syd \red{a}nd then come to ged{[}er{]}\\
\nmn{2}then trett and retrett and torn

\subsection{Lybens}
\tmn{Lebens di\\ {[}{]}nens de 2{[}bus{]}}
\nmn{a\textsuperscript{n}}After the end of þe t{[}ra{]}ce the\\
first iij forth and torne\footnote{Added in the bottom margin} whill the \\
\srcpg{64}
\red{S}ec{[}o{]}nd retrett iij bake then com to\\
ged{[}er{]} and ethir torne into oders\\
plas; \red{t}hen last ma{[}n{]} iij forth \& torn\\
\nmn{2}whill þe first retrett; \red{t}hen come to\\
ged{[}er{]} in suchwys as þay ded Afore\\
and ethir end in ther own place\\
\nmn{1}then trett and retrett and torn

\subsection{Aras}
\tmn{Aras de 2{[}obus{]}}\red{A}fter the end of the trace rak\\
both on way; \red{t}hen the first ma{[}n{]}\\
torne whill þe sec{[}o{]}nd retrett; \red{þ}en\\
\nmn{2}face to face rak \nmn{2}{[}con{]}t{[}ra{]}ry way; \red{t}hen\\
the sec{[}o{]}nd torne whill þe first\\
retrett; \red{t}hen trett and retrett\\
\nmn{2}at onys and come to ged{[}er{]} whith\\
obeysawnce; \red{t}hen torn bessily\\
to þ{[}r{]} Awn place Ayen; \red{t}hen e{[}ver{]}\\
{[}con{]}t{[}ra{]}ry other iij singlis and iij bak\\
\nmn{1}Ayen; \red{t}hen A flowre delice and\\
torne

\subsection{Eglamore}
\tmn{Eglamowr\\ de trib{[}us{]}}After the end of the tr{[}ace{]} 
\\ \ \\
\srcpg{65}
\red{T}he first thre forth þe \tmn{Double t{[}ra{]}}sec{[}o{]}nd þe\\
\nmn{3}same the 3\textsuperscript{d} the same; \red{t}hen the\\
first ma{[}n{]} outward on the left\\
shulder and goo behend þe 2\textsuperscript{d}\\
\nmn{3}the same the 3\textsuperscript{d} þe same; \red{t}hen\\
the first out þe 2\textsuperscript{d} out þe 3\textsuperscript{d} out

\subsection{New Year}
\tmn{New yer\\ de t{[}ri{]}b{[}us{]}}
\nmn{3\textsuperscript{a}}\red{A}fter the end of the t{[}ra{]}ce þe firste\\
ma{[}n{]} iij forth the 2\textsuperscript{d} þe same þe  3\textsuperscript{d}\\
\nmn{3}the same; \red{t}hen Al togeder halfe\\
 \tmn{A doble tr{[}ac{]}e} torne thre tymes; \red{t}hen the last\\
\nmn{3}thre forth the sec{[}o{]}nd the same þe\\
3\textsuperscript{d} þe same; \red{a}nd then al togedeer\\
\nmn{3}half torne\footnote{Marginal 3 could apply to either this dance or the next}

\subsection{Roye}
\tmn{Roye de 3 b{[}us{]}}\red{A}fter the end of þ{[}e{]}\\
trace the first man thre furth\\
\tmn{A doble t{[}ra{]}ce\\ \& iij bak i{[}n{]} þ{[}e{]} en{[}d{]}}the sec{[}o{]}nd the same þe iij the sam\\
\red{t}hen Al rak togeders forth; \red{t}hen\\
bak rak Ayen and All torn toged\\
\red{t}hen the first man move þen þe\\
sec{[}o{]}nd and\nmn{3} þe 3\textsuperscript{d} move \& þe first ma{[}n{]} \\
\srcpg{66}
\nmn{2} torne Abought; \red{t}hen þe sec{[}o{]}nd ma{[}n{]}\\
me{[}v{]}e; the first and þe \nmn{3}last meve\\
toged{[}er{]} \& the 2\textsuperscript{d} t{[}or{]}ne Abought the\\
þe last meve;\red{þ}en þe first \& 2\textsuperscript{d}\\
meve and al t{[}ur{]}ne Abowth toged{[}er{]}

\subsection{Orange}
\tmn{Oringe\\ de 3 b{[}us{]}}\red{A}fter the end of þe trace the\\
first iij forth; þe 2\textsuperscript{d} þe same þe\\
\tmn{A doble t{[}ra{]}ce\\ \& lok at þ{[}e{]} end\\ bak and al\\ to{[}ge{]}d{[}er{]} do obbey}3\textsuperscript{d} the same; \red{t}hen {[}þe{]} first man\\
throth þa{[}m{]} and go behend; \red{t}hen\\
the 2\textsuperscript{d} ma{[}n{]} iij forth þe 3\textsuperscript{d} þe sam\\
\nmn{3}þe first þe sam; and throth þa{[}m{]}\\
and goo behend; \red{t}hen 3\textsuperscript{d} man\\
thre forth the  fi{[}r{]}st þe same þe 2\textsuperscript{d}\\
þe sam \& the 3\textsuperscript{d} throth þa{[}m{]} \& goo\\
\nmn{3}behynd; \red{t}hen þe first iij bak þ{[}en{]}\\
\nmn{3}þe secn{[}o{]}d iij bak; \red{t}hen all toged{[}er{]}\\
thre forth and thre bake \& t{[}or{]}ne \\
\srcpg{67}
\subsection{Hawthorne}
\tmn{Hawthorne \\ de duob{[}us{]}} Thre singlis and thre Rakkys \&\\
a stop and torne; then þ{[}e{]} First 3\\
synglis and þre retrette þe ij\textsuperscript{de} e{[}con{]}tra\\
the same tyme þe 2\textsuperscript{de} thre singlis\\
\& thre retrette; the First e{[}con{]}tra the\\
same ty{[}m{]}; the first trett \& retrett\\
and torne then þe sec{[}o{]}nd þe same\\
;then face to face þe first make\\
A floure delice and þre retrette\\
the 2\textsuperscript{d} þe same; then at onys A\\
Flo{[}ur{]} delice and cu{[}m{]} to gedd{[}er{]}; then\\
þe horne horne pepy to gedd{[}er{]}\\
then at onys A long trett retrett\\
and torne

\subsection{Newcastle}
\tmn{Newcastell\\ de dubus}The first thre forth\\
\& torne whyls þe last retrett rek\\
\tmn{Suff}the last torne whils þe first retrett;\\
then þe first aflo{[}ur{]} delice þe 2\textsuperscript{e} þe\\
\tmn{a dubull t{[}ra{]}ce}same; then thre singlis a þ\textsuperscript{e} {[}con{]}tra to od{[}er{]}; \\
\srcpg{68}
And in the same wis {[}con{]}tr{[}ar{]}y agayn \&\\
come togedd{[}er{]}; then trett retrett and\\
torne;

\subsection{Damesyn}
\tmn{Damesyn\\ de 3b{[}us{]}}All togedd{[}er{]} thre singlis w{[}ith{]} {t{[}ra{]}ce}\footnote{"trace" appears in the margin, has been moved into the text}\\
a stope \textsuperscript{+} iij retrett \& thre forth and\\
so ij tymys; then at þe iij\textsuperscript{d} t{[}ra{]}ce 3\\
singlis w{[}ith{]} a stope; then þe first \&\\
þe last a holl torne the medyll A\\
halfe torn þe same ty{[}m{]}; the first\\
and the last forth right þe medill\\
{[}con{]}tr{[}ar{]}y them and the end þay two\\
changs place whil; þe medill torn\\
hy{[}m{]} to þam; the first to þe last;\\
manys place; þe last to his place\\
all togedd{[}er{]} 2 whill þe medyll\\
goo throth þame; the first lepe\\
all togedd{[}er{]} lepe then þe last;\\
lepe; þen the medyll; þen torn\\
all at onys; the first \& þe last; \\
\srcpg{69} 
Forth right þe medill {[}con{]}tr{[}ar{]}y hem\& \\
in þe end þay chang place whils\\
the medyll torn toward þam and\\
in the sam wys agayn savyng in\\
in the end ev{[}er{]}y ma{[}n{]} kep his one\\
place; then þe last to þe first ma{[}n{]}is\\
place and first to þe last manys\\
place the medill come to w{[}ith{]} thre\\
singlis

\subsection{Rawty}
\tmn{Rawty\\ de 2bus}Trett retrett togedd{[}er{]}s\\
then thre singlis w{[}ith{]} a stope and\\
thre retrett w{[}ith{]} a stop both togedd{[}er{]}\\
\tmn{A dubel {[}trace{]}}in þe sam wis agayn; then trett\\
and retrett and dep{[}er{]}t þe first\\
forth rygh; aflo{[}ur{]} delice and cum\\
togedd{[}er{]} and athir rak {[}con{]}tr{[}ar{]}y to;\\
othir agay aflo{[}ur{]} delice and cu{[}m{]}\\
togedder \\
\srcpg{70}

\subsection{Temperance}
\tmn{Temp{[}er{]}ans\\ de 3b{[}u{]}} All trett and retrett thre singlis\\
w{[}ith{]} a stop then þe first and þ{[}e{]} me=\\
dyll togedd{[}er{]} retrett; while þe\\
last goth forth and cu{[}m{]} togedd{[}er{]}\\
Agayn in the same forme sav=\\
yng in the end wher þe laste;\\
\tmn{Trace}dep{[}er{]}tytt þe secn{[}o{]}d shall þe thred\\
the first shall; then ev{[}er{]}y man\\
togedd{[}er{]} two dubles on þe left\\
fott and halfe a torn; then ev{[}er{]}y\\
man the same {[}con{]}tr{[}ar{]}y wyes agayn\\
then all togedd{[}er{]} trett retrett\\
and thre Rakke then all toged{[}er{]}\\
Aflo{[}ur{]} delice; then þe first ma{[}n{]}\\
throw whils þe tothir two cha=\\
nge place; then þe first 3 forth\\
w{[}ith{]} a torn þe 2\textsuperscript{d} þe same þe 3 þe\\
same w{[}ith{]} out a torn then the \\
\srcpg{71} 
first lepe the secn{[}o{]}d lepe þe last\\
lepe; then all torn w{[}ith{]} a rest\\
in the meddys

\subsection{Northumberland}
\tmn{Northhu{[}m{]}b\\land de 3{[}bus{]}}Togedd{[}er{]} thre si{[}n{]}g=\\
lis w{[}ith{]} a stop trett and retrett;\\
then þe first \sout{þ} ma{[}n{]} torn in hys\\
\tmn{Trace}own place whill the last torn\\
the medyll abowytt; togedd{[}er{]}\\
agayn w{[}ith{]} thre singlis and a stop\\
trett and retrett; then halfe\\
a torn and torn agayn 3 toge=\\
dd{[}er{]} then agay w{[}ith{]} thre singlis\\
and a stope \& trett and retrett\\
then þe first; torn þe meddyll\\
abowt whils þe last ma{[}n{]} torn\\
in his own place; then ev{[}er{]}y\\
man a mevyng; then þe first\\
and the last torn owtward go=\\
yng forth vj singlis the 2\textsuperscript{d} forth \\
\srcpg{72}
right torn all face to face and þ\textsuperscript{e}\\
thred brayll þe toþ\textsuperscript{e} two cu{[}m{]} to hy{[}m{]}\\
and goo to þ\textsuperscript{e} place agay{[}n{]}; then\\
þe 3\textsuperscript{d} and þe 2\textsuperscript{d} brayll the;\\
meddyll cu{[}m{]} to þem and go to\\
his awn place agayn; then\\
the first and þe last lepe to\\
gedd{[}er{]} þe meddyll lepe Alone;\\
then þe meddyll throw whils\\
othir two hym t{[}ur{]}nyg\\
all face to face; All at onys aflo{[}ur{]}\\
delice the first and þe meddyl\\
rak tell þay mett whill þe last;\\
retrett; aflo{[}ur{]} delice at onys þe\\
meddell and þe last rake whil\\
þay mett while the first\\
retrett than all cu{[}m{]} togedder

\newpage

    \section{A Section}
    \lipsum[1]
    
    % bib stuff
    \nocite{*}
    \addtocontents{toc}{\protect\vspace{\beforebibskip}}
    \addcontentsline{toc}{section}{\refname}    
    \bibliographystyle{plain}
    \bibliography{../Bibliography}
\end{document}
